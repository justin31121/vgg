\documentclass[10pt]{beamer}


\usetheme{tufi}
\usepackage{wasysym}
\usepackage{ucs}
\usepackage[ngerman]{babel}
\usepackage[latin1]{inputenc}
\usepackage{amsmath,amsfonts,amssymb}
\usepackage{graphicx}
\usepackage[T1]{fontenc}
\usepackage{verbatim}
\usepackage[babel,german=quotes]{csquotes}

%========================================================================================

\pdfinfo
{
  /Title       (Bilderkennung mit Vgg)
  /Creator     (TeX)
  /Author      (Justin Schartner)
}

\title{Proseminar \enquote{Convolutional Neural Networks - Methoden und Anwendungen}}
\subtitle{Bilderkennung mit Vgg}
\author{Justin Schartner}
\date{\today}

%========================================================================================

\begin{document}

\frame{\titlepage}

\AtBeginSection[]
{
  \frame<handout:0>
  {
    \frametitle{Struktur}
    \tableofcontents[currentsection,hideallsubsections]
  }
}

\AtBeginSubsection[]
{
  \frame<handout:0>
  {
    \frametitle{Struktur}
    \tableofcontents[sectionstyle=show/hide,subsectionstyle=show/shaded/hide]
  }
}

\newcommand<>{\highlighton}[1]{
  \alt#2{\structure{#1}}{{#1}}
}

\newcommand{\icon}[1]{\pgfimage[height=1em]{#1}}

%INHALT=============================================================================
%=========================================
\section{Einleitung}
\begin{frame}
	\frametitle{Thema}
	\begin{block}{Problemstellung}Gibt es CNN-Architekturen, welche Bilder noch besser, als schon bekannte Architekturen klassifizieren k\"onnen?
	\end{block}
	\begin{block}{L\"osung}Durch die Erh\"ohung der Tiefe eines CNN verspricht man sich genauere Aussagen \"uber Bilder machen zu k\"onnen.
	\end{block}
	\begin{block}{Ergebnis}Die VGG-Architektur hat bewiesen, dass die Tiefe eines CNN, eine ausschlaggebende Komponente hinsichtlich der Bilder-Klassifizierung ist.
	\end{block}
\end{frame}

\begin{frame}
	\frametitle{Gliederung}
	\begin{itemize}
	\setlength\itemsep{1em}
	\item Was ist VGG? \textbf{Allgemeines}
	\item Wie funktioniert VGG, was macht es besonders? \textbf{Architektur}
	\item Wie trainiert man ein VGG-net, was ist wichtig? \textbf{Training}
	\item Wie implementiert man ein VGG-net? \textbf{Beispiel}
	\item Wie gut ist VGG? \textbf{Bewertung}
	\end{itemize}
\end{frame}
%=========================================
\section{Allgemeines}
\begin{frame}
  \frametitle{Eckdaten}
	\begin{itemize}
		\setlength\itemsep{1em}
		\item \textbf{V}isual \textbf{G}eometry \textbf{G}roup
		\item Department of Engineering Sciecne, University of Oxford
		\item Karen Simonyan und Andrew Zisserman
	 	\item \textbf{Ver�ffentlicht}: 4 Sep 2014
		\item \textbf{Letzte �nderung}: 10 Apr 2015
	\end{itemize}
\end{frame}

\begin{frame}
	\frametitle{Idee}
	\begin{block}{Steigerung der Genauigkeit durch:}
	\begin{itemize}
		\setlength\itemsep{1em}
		\item Steigerung der Tiefe, des CNNs
		\item Schachtelung von Convolutional-Layer-Bl\"ocken
		\item Einsatz von kleinen 3x3-Filtern und einer Stride von 1
	\end{itemize}
	\end{block}
\end{frame}

\begin{frame}
	\frametitle{Aufgaben/ Einsatz}
	\begin{block}{Bilderkennung}
		\begin{itemize}
			\setlength\itemsep{1em}
			\item Imagenet
			\item Pneumoina Image
			\item Deep Facial Emotion Recognition
			\item Plankton Classification
			\item Plant Image Classification
			\item ...
		\end{itemize}
	\end{block}
\end{frame}

%=========================================
\section{Architektur}		
\begin{frame}
	\frametitle{Architekturarten}
		\begin{center}
			{\tiny
			\begin{tabular}{| c | c | c | c | c | c |}
				\hline
				\multicolumn{6}{|c|}{ConvNet Configuration} \\
				\hline
				A & A-LRN & B & C & D & E \\
				\hline
				\vtop{\hbox{\strut 11 weight}\hbox{\strut layers}} &
				\vtop{\hbox{\strut 11 weight}\hbox{\strut layers}} &
				\vtop{\hbox{\strut 13 weight}\hbox{\strut layers}} &
				\vtop{\hbox{\strut 16 weight}\hbox{\strut layers}} &
				\vtop{\hbox{\strut 16 weight}\hbox{\strut layers}} &
				\vtop{\hbox{\strut 19 weight}\hbox{\strut layers}} \\
				\hline \hline
				\multicolumn{6}{|c|}{input (224 x 224 RGB-image)} \\
				\hline
				\vtop{\hbox{\strut conv3-64}} &
				\vtop{\hbox{\strut conv3-64}\hbox{\strut \textbf{LRN}}} &
				\vtop{\hbox{\strut conv3-64}\hbox{\strut \textbf{conv3-64}}} &
				\vtop{\hbox{\strut conv3-64}\hbox{\strut conv3-64}} &
				\vtop{\hbox{\strut conv3-64}\hbox{\strut conv3-64}} &
				\vtop{\hbox{\strut conv3-64}\hbox{\strut conv3-64}} \\
				\hline
				\multicolumn{6}{|c|}{maxpool} \\
				\hline
				\vtop{\hbox{\strut conv3-128}} &
				\vtop{\hbox{\strut conv3-128}} &
				\vtop{\hbox{\strut conv3-128}\hbox{\strut \textbf{conv3-128}}} &
				\vtop{\hbox{\strut conv3-128}\hbox{\strut conv3-128}} &
				\vtop{\hbox{\strut conv3-128}\hbox{\strut conv3-128}} &
				\vtop{\hbox{\strut conv3-128}\hbox{\strut conv3-128}} \\
				\hline
				\multicolumn{6}{|c|}{maxpool} \\
				\hline
				\vtop{\hbox{\strut conv3-256}\hbox{\strut conv3-256}} &
				\vtop{\hbox{\strut conv3-256}\hbox{\strut conv3-256}} &
				\vtop{\hbox{\strut conv3-256}\hbox{\strut conv3-256}} &
				\vtop{\hbox{\strut conv3-256}\hbox{\strut conv3-256}\hbox{\strut \textbf{conv1-256}}} &
				\vtop{\hbox{\strut conv3-256}\hbox{\strut conv3-256}\hbox{\strut \textbf{conv3-256}}} &
				\vtop{\hbox{\strut conv3-256}\hbox{\strut conv3-256}\hbox{\strut conv3-256}\hbox{\strut \textbf{conv3-256}}} \\
				\hline
				\multicolumn{6}{|c|}{maxpool} \\
				\hline
				\vtop{\hbox{\strut conv3-512}\hbox{\strut conv3-512}} &
				\vtop{\hbox{\strut conv3-512}\hbox{\strut conv3-512}} &
				\vtop{\hbox{\strut conv3-512}\hbox{\strut conv3-512}} &
				\vtop{\hbox{\strut conv3-512}\hbox{\strut conv3-512}\hbox{\strut \textbf{conv1-512}}} &
				\vtop{\hbox{\strut conv3-512}\hbox{\strut conv3-512}\hbox{\strut \textbf{conv3-512}}} &
				\vtop{\hbox{\strut conv3-512}\hbox{\strut conv3-512}\hbox{\strut conv3-512}\hbox{\strut \textbf{conv3-512}}} \\
				\hline
				\multicolumn{6}{|c|}{maxpool} \\
				\hline
				\vtop{\hbox{\strut conv3-512}\hbox{\strut conv3-512}} &
				\vtop{\hbox{\strut conv3-512}\hbox{\strut conv3-512}} &
				\vtop{\hbox{\strut conv3-512}\hbox{\strut conv3-512}} &
				\vtop{\hbox{\strut conv3-512}\hbox{\strut conv3-512}\hbox{\strut \textbf{conv3-512}}} &
				\vtop{\hbox{\strut conv3-512}\hbox{\strut conv3-512}\hbox{\strut \textbf{conv3-512}}} &
				\vtop{\hbox{\strut conv3-512}\hbox{\strut conv3-512}\hbox{\strut conv3-512}\hbox{\strut \textbf{conv3-512}}} \\
				\hline
				\multicolumn{6}{|c|}{maxpool} \\
				\hline
				\multicolumn{6}{|c|}{FC-4096} \\
				\hline
				\multicolumn{6}{|c|}{FC-4096} \\
				\hline
				\multicolumn{6}{|c|}{FC-1000} \\
				\hline
				\multicolumn{6}{|c|}{soft-max} \\
				\hline
			\end{tabular}
		}
		\end{center}
\end{frame}

\begin{frame}
	\frametitle{Architekur}
	\begin{block}{Convolutional Layers}
		\begin{itemize}
			\item receiptive field: 3x3, 1x1
			\item activation: RelU, stride: 1, padding: 1, channels: 64, 128, 512, 512
		\end{itemize}
	\end{block}
	\begin{block}{Pooling Layer}
		\begin{itemize}
			\item Max-Pool
			\item field: 2x2, stride: 2
		\end{itemize}
	\end{block}
	\begin{block}{Fully-Connected Layers}
		\begin{itemize}
			\item activation: RelU
			\item channels: 4096, 4096, 1000
		\end{itemize}
	\end{block}
	\begin{block}{Softmax Layer}
	\end{block}
\end{frame}

\begin{frame}
	\frametitle{Convolutional Layers}
	\begin{itemize}
		\item 
	\end{itemize}
\end{frame}

\begin{frame}
	\frametitle{Hidden Layers}
\end{frame}

\begin{frame}
	\frametitle{Fully Connected Layers}
\end{frame}

\begin{frame}
	\frametitle{Softmax}
\end{frame}

%=========================================
\section{Training}
\begin{frame}
	\frametitle{Title}
	\framesubtitle{Subtitle}
	\begin{block}{LaTex}
	\begin{itemize}
		\item Ein Punkt
		\item Zwei Punkte
	\end{itemize}
	\end{block}
\end{frame}

%=========================================
\section{Beispiel}
\begin{frame}
	\frametitle{Vgg initialisieren}
\end{frame}

\begin{frame}
	\frametitle{Vgg laden}
\end{frame}

\begin{frame}
	\frametitle{Vgg trainieren}
\end{frame}

\begin{frame}
	\frametitle{Vgg benutzen}
\end{frame}

%=========================================
\section{Bewertung}
\begin{frame}
	\frametitle{Komplexit�t}
\end{frame}

\begin{frame}
	\frametitle{Performance}
\end{frame}

\begin{frame}
	\frametitle{Vergleich}
\end{frame}

%==========================================
\section{Ausblick}
\begin{frame}
	\frametitle{Ausblick}
\end{frame}


\end{document}
